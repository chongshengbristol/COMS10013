%ps1.tex
%notes for the course Probability and Statistics COMS10011 
%taught at the University of Bristol
%2019_20 Conor Houghton conor.houghton@bristol.ac.uk

%To the extent possible under law, the author has dedicated all copyright 
%and related and neighboring rights to these notes to the public domain 
%worldwide. These notes are distributed without any warranty. 

\documentclass[11pt,a4paper]{scrartcl}
\typearea{12}
\usepackage{graphicx}
%\usepackage{pstricks}
\usepackage{listings}
\usepackage{color}
\lstset{language=C}
\usepackage{fancyhdr}
\pagestyle{fancy}
\lhead{\texttt{github.com/coms10013/2022\_23} and  \texttt{coms10013.github.io}}
\lfoot{COMS10013 - ws1 - Conor}
\begin{document}

\section*{COMS10013 - Analysis - WS1}

\subsection*{Useful facts}

\begin{itemize}

\item polynomials: $dx^n/dx=nx^{n-1}$
\item special function: $d\sin{x}/dx=\cos{x}$, $d\cos{x}/dx=-\sin{x}$, $d\exp{x}/dx=\exp{x}$, $d\log{x}/dx=1/x$.
\item product rule:
$$\frac{d}{dx}uv = \frac{du}{dx}v+u\frac{dv}{dx}$$
\item quotient rule:
$$\frac{d}{dx}\frac{u}{v}=\frac{\frac{du}{dv}v-u\frac{dv}{dx}}{v^2}$$
\item chain rule:
$$\frac{d}{dx}u(v(x))=\frac{du}{dv}\frac{dv}{dx}$$
\item reminder regarding exponentials and logs: $\exp{\log{x}}=x$ and $\log{a^b}=b\log{a}$.
\item gradients for $f(x,y)$; $\nabla{f}=(f_x,f_y)$ where $f_x=\partial f/\partial x$.
\item the Hessian
$$H(f)=\left(\begin{array}{cc}f_{xx}&f_{xy}\\f_{yx}&f_{yy}\end{array}\right)$$
\item the determinant of a matrix is equal the multiple of its eigenvalues, the trace is the sum.

\end{itemize}

\subsection*{Questions}

These are the questions you should make sure you work on in the workshop.

\begin{enumerate}

\item Differentiate the following functions with respect to $x$;
\begin{enumerate}
\item $3x^2$
\item $(x+2)^2$
\item $ae^{cx}$ where $a$ and $c$ are constants.
\item $\exp{x^2}$
\item $\sin^2{x}+\cos^2{x}$
\item $\cos^2{x}-\sin^2{x}$
\item $\exp{1/x}$
\end{enumerate}

\item Find the local minima and maxima of $y=x^5-3x^2+6$.
 
\item Find the partial derivatives of $z(x,y)=5x^2y+2x\sin{y}$.

\item Find the gradient of $z(x,y)=(x+y^2)^2$.

\end{enumerate}

\subsection*{Extra questions}

These are extra questions you might attempt in the workshop or at a later time.

\begin{enumerate}

\item Differentiate $x^x$ with respect to $x$.
\item The function $z(x, y) = x^2 + y^2 + 2x - 3y$ has a global minimum. Find this by taking
the gradient and searching for the point where the gradient is zero.
\item Check that this point you found really is a minimum by computing the Hessian of the
function at this point, and checking that it is positive definite, that is, all eigenvalues are positive.


\end{enumerate}

\end{document}